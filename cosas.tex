\documentclass[11pt, a4paper]{article}
\usepackage{anysize}



%==============================================
% Para agregar una imagen
% \pgfimage[width=0.9\textwidth]{imgs/cproject.png}
%
% Para agregar código
% \begin{lstlisting}[opts]
% \end{lstlisting}

%\begin{figure}[!h]
%\centering
%\pgfimage[width=0.9\textwidth]{imgs/cproject.png}
%\caption{Modelo de Quadrotor}
%\label{fig:QuadRotor}
%\end{figure}

%==============================================
\marginsize{3cm}{2cm}{1cm}{1cm}

\usepackage[spanish]{babel}
\usepackage{graphicx}
\usepackage{fancyhdr}
\usepackage[T1]{fontenc}
\usepackage{lmodern}
\usepackage{color}
\usepackage{listings}
\usepackage{hyperref}
\usepackage{pgf}

%=================================================
%=================================================

\begin{document}

%Cabeceras y pies de p·gina
\pagestyle{fancy}
\lhead{\title}
\chead{}
\rhead{}

\lfoot{}
\cfoot{}
\rfoot{\thepage}

%=================================================
%=================================================
%\tableofcontents
%\newpage

%=================================================
\section{Proyecto}
Falta en alguna parte poner objetivos, para que o hacemos, que esperamos poder lograar y para que se pued eusar todo esto eque estamos haciendo.
%-------------------------------------------------------------
\subsection{Work Breakdown structure}

Esta es una referencia a lo que es un WBS \ref{ref:WBS}

\begin{figure}[!h]
\centering
\pgfimage[width=0.9\textwidth]{imgs/wbs.png}
\caption{WBS: Quadrotor Porject}
\label{fig:QuadRotor}
\end{figure}

%-------------------------------------------------------------
\subsection{Diagrama en bloques}

Aca deberia ir un diagrama de bloques en donde esten un PC por parte del segmento tierra y uno o m�s quadrotores. Deberia estar bueno qeu se vea que la comunicaci�n es inalambrica y que en ppo el sistema esta pensando para comunicaci�n Uplink/Downlink con GS y tambi�n CrossLink entre quads. 

%=================================================
\section{La aeronave}

%-------------------------------------------------------------
\subsection{Diagrama en bloques}

Esta faltando una imagne en donde se pueda ver que a aeronave esta compuesta por una bateria, 4 motores, 4 ESC, 4 helices, una computadora de abordo con comunicaci�n a tierra y cross


%-------------------------------------------------------------
\subsection{Motherboard - Computadora de vuelo}

\subsubsection{Hardware}

En la figur \ref{fig:Motherboard} se puede observar un diagrama de bloques de la placa madre de la aeronave en su primera versi�n:
\begin{figure}[!h]
\centering
\pgfimage[width=0.7\textwidth]{imgs/MotherBoard.png}
\caption{Motherboard}
\label{fig:Motherboard}
\end{figure}

\subsubsection{Arquitectura de Software}

%=================================================
\section{Sistema de comunicaciones}

\subsection{Requerimientos}
Se necesit� implementar un subsistema de comunicaciones que sea confiable y que cumpla las siguientes carateristicas:

\begin{itemize}
\item Comunicaci�n inalambrica entre nodos
\item Debe poder transportar diferentes tipos de datos en ambos sentidos por un mismo canal.
\end{itemize}

\subsection{Modelo}

El sistema se modela en 4 capas. A continuaci�n se describe cada una de estas capas.

\begin{figure}[!h]
\centering
\pgfimage[width=0.7\textwidth]{imgs/modelo.png}
\caption{Modelo del sistema de comunicaciones}
\label{fig:Coms}
\end{figure}


\subsubsection{F�sica}
En enlace f�sico entre dispositios se hace mediante radios Xbee que implementan el protocolo ZigBee. Por el momento las radios Xbee se usan en modo transparente y son una conexi�n inalambrica de UARTS.
No se utiliza el modo de direccionamiento de los XBee. De haber direccionamiento se implementar� en capas superiores.

\subsubsection{Enlace}

Las comunicaciones se realizan de UART a UART. La misma transporta siempre a de a 1 byte de informaci�n a 115200bps. En esta capa se debe implementar un sistema de buffer que permite almacenar los datos recibidos mientras se esta realizando alguna otra tarea.
En la implementaci�n del BSP de la UART para el LPCXpresso se utilizar� DMA como un m�todo de double buffering.

\subsubsection{Transporte}

Esta capa es la encarga de determinar que los datos que lleguen y salgan cumplan con un formato preestablecido. De esta manera se pueden detectar errores en la comunicaci�n y tambi�n permite un primer nivel de parseo de datos ya que permite distinguir diferentes tipos de paquetes.

Un paquete de en esta capa tiene la siguiente forma:

\begin{figure}[!h]
\centering
\pgfimage[width=0.9\textwidth]{imgs/mensaje.png}
\caption{Mensaje}
\label{fig:Mensaje}
\end{figure}

De donde se puede ver que los paquetes tienen una carga �til (mensajes de la capa de aplicaci�n) del longitud variable y su tama�o puede variar entre 0 y 255 bytes. Luego la capa de aplicaci�n es la encargada de dividir el paquete si el tama�o m�ximo no es suficiente. 

La implementaci�n de esta capa es la responsable de determinar que los paquetes lleguen en forma y se envien en forma, es decir:

\begin{itemize}
\item verificar las direcciones de source y destination
\item verificar el numero de secuencia o timestamp si es necesario
\item verificar el checksum
\end{itemize}

El campo de tipo de dato se encarga de determinar que tipo de datos se transporta en la carga util. Luego esto es redirigido seg�n la aplicaci�n a distintas secciones de la aplicaci�n. Esto se explica en la siguiente subsecci�n.

\subsubsection{Aplicaci�n}

Los paquetes pueden tener 255 diferentes tipos de datos. Los implementados al d�a de hoy son:

\begin{itemize}
\item \textbf{SYSTEM:}Mensajes de control del sistema. Permite configurar el quadrotor e informar sobre acciones a tierra.
\item \textbf{CONTROL:} Mensajes de control de vuelo, cambio de posici�n, acciones, etc.
\item \textbf{DEBUG:}Mensajes de baja prioridad. Deben ser interpretados en bulk como caracteres ASCII y mostrados en una consola en el lado de TIERRA. En el lado del QUAD es opcional implementar un parser
\item \textbf{TELEMETRY:}Paquete de datos de telemetr�a.
\end{itemize}

%=================================================
\section{Segmento de tierra}

El segmento de tierra esta compuesto por una aplicaci�n de comunicaciones principal. La misma, denominada RadioServer, es la encargada de enviar y recibir datos con la o las aeronaves. Dado que a traves del mismo canal se necesitaban enviar y recibir diferentes tipos de datos, y en general de diferentes tipos de aplicaciones ejecutandose, por ejemplo, en diferentes computadores. 

Se implemento un servidor escrito en Python que permite redireccionar a puertos TCP cada tipo de datos enviando todo lo recibido desde la aeronave a los diferentes clientes y viceversa.

Un esquema de este subsistema se muestra en la figura \ref{fig:ground_coms}.

\begin{figure}[!h]
\centering
\pgfimage[width=0.7\textwidth]{imgs/ground_coms.png}
\caption{Comunicacion en tierra}
\label{fig:ground_coms}
\end{figure}

Las ventajas encontradas en esta implementaci�n son: versatilidad ya que la comunicacion por sockets esta ampliamente desarrolladas en la mayoria de los entornos de trabajo como ser: MATLAB y Simulink, C++, Python, Simualdores de vuelo, etc., multiples clientes pueden recibir la misma informaci�n y visualizarla o interpretarla de forma diferente, los clientes pueden estar distibuidos en diferentes computadoras dentro de una LAN o internet.





\end{document}